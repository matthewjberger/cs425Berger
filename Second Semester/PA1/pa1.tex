\documentclass{scrreprt}
\usepackage{graphicx}
\usepackage{listings}
\usepackage{underscore}
\usepackage[bookmarks=true]{hyperref}
\usepackage[utf8]{inputenc}
\usepackage[english]{babel}
\usepackage{subfigure}
\usepackage{hyperref}
\usepackage[table,xcdraw]{xcolor}
\setlength{\footskip}{140pt}
\graphicspath{ {images/} } 
\hypersetup{
    bookmarks=false,    % show bookmarks bar?
    pdftitle={Design and Project Management Update},    % title
    pdfauthor={Matthew Berger},                     % author
    pdfsubject={TeX and LaTeX},                        % subject of the document
    pdfkeywords={TeX, LaTeX, graphics, images}, % list of keywords
    colorlinks=true,       % false: boxed links; true: colored links
    linkcolor=blue,       % color of internal links
    citecolor=black,       % color of links to bibliography
    filecolor=black,        % color of file links
    urlcolor=purple,        % color of external links
    linktoc=page            % only page is linked
}%
\def\myversion{1.0 }
\date{}
%\title
\usepackage{xcolor,colortbl}
\newcommand{\mc}[2]{\multicolumn{#1}{c}{#2}}
\definecolor{LightGray}{gray}{0.95}
\definecolor{Gray}{gray}{0.85}
\definecolor{LightCyan}{rgb}{0.88,1,1}

\newcolumntype{a}{>{\columncolor{Gray}}p{0.3\linewidth}}
\newcolumntype{b}{>{\columncolor{white}}p{0.7\linewidth}}
\newcolumntype{d}{>{\columncolor{LightGray}}p{0.3\linewidth}}
\newcolumntype{e}{>{\columncolor{white}}p{0.6\linewidth}}
\newcolumntype{f}{>{\columncolor{LightGray}}p{0.2\linewidth}}
\newcolumntype{g}{>{\columncolor{Gray}}p{0.2\linewidth}}
\newcolumntype{h}{>{\columncolor{white}}p{0.8\linewidth}}

\begin{document}

\begin{flushright}
    \rule{16cm}{5pt}\vskip1cm
    \begin{bfseries}
        \Huge{PROJECT MANAGEMENT SPECIFICATIONS}\\
        \vspace{1.6cm}
        for\\
        \vspace{1.6cm}
        $NAVATAR$\\
        \vspace{1.6cm}
        \LARGE{Version \myversion approved}\\
        \vspace{1.6cm}
        Prepared by Matthew Berger, Liam Gomez, and Connor Parkinson\\ 
        \vspace{1.6cm}
        Advised by Professor Eelke of the\\UNR CSE Department\\
        \vspace{1.6cm}
        \today\\
    \end{bfseries}
\end{flushright}

\tableofcontents

\chapter{Abstract}
Navatar is an indoor navigation system for visually impaired students. Unlike existing systems that rely on expensive equipment, Navatar only requires an Android smartphone. The phone’s GPS, accelerometer, and compass are used to approximate a user’s location and movements and the phone’s speaker is used to provide auditory directions to rooms within a building. Buildings on University campuses often have the GIS maps and the design required for Navatar’s accelerometer and compass based localization, which is why students are the intended audience for the project. With the application’s open source status and modular design, crowdsourcing efforts can be used to integrate maps for buildings on any campus. The application is still under development and Team 6 is working on project management and scheduling for the implementation of a variety of new features alongside other open source contributors.


\chapter{Project Updates and Changes}
Team 6 has been working on cleaning up the existing code base to better facilitate the implementation of the planned features while also debugging issues encountered with the existing code. The team has recently added additional priority to fixing bugs with the existing maps and map creation functionality as their advisor, Eelke Folmer, informed them a university has expressed interest in the development of Navatar. Efforts are also being made to automate setup of the development environment to allow other open source contributors to easily assist with improving Navatar.
It became evident after the design and prototype stage that some of the team's ideas were better in theory than in implementation. One such example was the gesture-detection as a form of interaction with the application. Gestures that are context-aware lead to the same application interactions eliciting separate and highly distinct behaviors that may lead to unexpected results for the user. The context of the application may be difficult to discern for a blind user. In this sense, gesture-detection should be superseded and brought to obsolescence by voice commands. We changed our design to use Google voice to implement the actual voice recognition and voice commands. Simply speaking to the application through a microphone on a headset or set of ear-buds will allow for more accurate control and potentially limitless interactions with the app that gestures simply would not allow for. There are only a limited number of gestures, and the overlap with existing accessibility applications would make the usability ineffective. Voice commands are the chosen options here. After designing, prototyping, and testing the GPS features we found that reliably detecting a user’s location indoors was difficult. It was determined that the real use-case for autolocation would be when a user enters a building, which allows for the best location accuracy using GPS. The subsequent use cases including the user leaving the building or navigating between rooms inside the building are made possible using an Android feature that allows for getting the last known location of the smartphone.The autolocation functionality is done using Geo-Fencing currently, and can be improved by using additional location data sources such as WiFi or cellular towers. As for navigation history, it is a nearly complete and functional feature barring the additional planned functionalities of allowing users to reverse the previously navigated route and collecting data on which routes are completable. Additional focus for improving the feature is allowing users to navigate the selections with easily spoken voice commands. Overall there is much to work on and many features to develop with the overarching task of refactoring the codebase to improve the quality of the existing code.

\chapter{Project Deliverables}
This team’s primary focus is implementing new features that enhance the accuracy and usability of the existing application. Each team member will be responsible for the development and implementation of their respective component outlined below.

\section{Route History - Liam Gomez}
This component is responsible for logging and storing a user’s previously taken routes. Storing the previous taken routes allows for a route repeat feature to be implemented, where a user has the option to view their route history and start a new navigation from that history. Storing route history also allows for a user to a reverse a given navigation after reaching their destination.

\subsection{User Stories}
\begin{enumerate}
	\item As a user, after opening Navatar, I should be given an option that takes me to the route history page so I can view my route history.
	\item As a user, on the route history page, I should be able to select a route from my history so I can repeat that navigation.
	\item As a user, after successfully completing a navigation, I should be prompted if I would like to reverse that route so I can navigate back to my origin.
	\item As a user, on the route history page, I would like to avoid having failed or incomplete navigations saved to my history so I can avoid repeating them.
	\item As a researcher, I should have access to the route history logs in a parsable format for analytical purposes.
\end{enumerate}

\section{Geofencing - Connor Parkinson}
This component is responsible for the automatic selection of the campus and building based on the user’s current location. This component will use a combination of the smartphones GPS, cellular, and Wi-Fi to provide locational data. This locational data will then be compared with the known geographical boundaries of available campuses and buildings.

\subsection{User Stories}
\begin{enumerate}
	\item As a user, I want Geofencing so that I don't have to select the campus I am on.
	\item As a user, I want Geofencing functionalities to be active when I open the app.
	\item As a user, I want to revert to existing campus and or building selection if Geofencing fails (is not accurate enough or times out).
	\item As a user, I want Geofencing so that I don't have to select the building I want to navigate within.
	\item As a user, I want Geofencing to select at least the campus even when accuracy is not high enough to select the building.
\end{enumerate}

\section{Gesture-Based Controls - Matthew Berger}
This component is responsible for providing an additional input method to the user that is outside the existing Google Talkback integration. For a user with visual impairments navigating content heavy menus and input fields can be quite tedious. To improve this, a gesture based input system will be implemented and integrated with the existing user interface. Gesture based input will allow the user to make gestures on their devices screen that control the applications behavior.

\subsection{User Stories} [TODO]
\begin{enumerate}
	\item As a user,.
	\item As a user,.
	\item As a user,.
	\item As a user,.
	\item As a user,.
\end{enumerate}

\chapter{Project Assignments and Schedule}
[TODO]

\chapter{Project Monitoring and Risk}

Following the updates to the requirements and design of the planned features, the team developed a schedule and timeline to ensure project success. The team delegated subsystems, responsibilities, and workload by having each team member volunteer for pre-established subsections and responsibilities based on their experience and interests. Project tasks were outlined on the scrum board with corresponding time estimates and the team developed the schedule with both realistic and stretch goals for what is planned to be accomplished. Weekly meetings have been established which allows for all team members to hold each other accountable for their assigned tasks and for the team to reallocate time for tasks which could be deemed more complex than originally anticipated. A Gantt chart was considered for outlining each scheduled task and dependencies, but the team deemed it unnecessary due to the independent design of each planned feature.

\section{Risk Management Strategies}
\begin{enumerate}
	\item[] \textbf{Sudden Increase in Requirements or Development time}
	\item Mitigating the risk of sudden requirement or time increases has been mitigated by thoroughly establishing the requirements and design with a corresponding schedule that allows for flexible time allocation.
	\item[] \textbf{Productivity Issues}
	\item By having a realistic schedule and frequent team meetings, the team members are held accountable for the progress expected to be achieved at every step of the development process.

	\item[] \textbf{Scheduling Issues}
	\item Establishing frequent team meetings is possible by having all team members commit early in the development process to certain times and dates with sufficient time to prioritize and plan.

	\item[] \textbf{Bugs with the Existing Codebase}
	\item The team understands that issues can arise when working with existing code which is why the planned features have been designed to be as modular as possible to avoid conflicts with existing code. The team has also allotted time for debugging essential functionalities.
	\item[] \textbf{Development Environment Consistency}
	\item Developing an application for Android involves a multitude of version specific dependencies which can be difficult to maintain and resolve conflicts between. The team is using Vagrant to configure a standard which is being used by all team members and will be suggested for all other open source contributors.

	\item Risk Management Item
	\item Risk Management Item
	\item Risk Management Item
	\item Risk Management Item
	\item Risk Management Item
\end{enumerate}

\chapter{Team Contributions}
	\section{Matthew Berger}
		\begin{itemize}
			\item Created LaTeX document
			\item Setup source and development environment controls
			\item Completed project assignments and schedule
			\item Created gesture subsystem user stories
			\item Completed five of the ten risk management strategies
		\end{itemize}
	\section{Connor Parkinson}
		\begin{itemize}
			\item Completed the abstract
			\item Outlined the project design and requirement updates
			\item Established the project monitoring plan
			\item Completed five of the ten risk management strategies
		\end{itemize}
	\section{Liam Gomez}
		\begin{itemize}
			\item Outlined the project deliverables
			\item Organized user stories on the team’s ZenBoard
			\item Compiled the references.
		\end{itemize}
	\section{All Team Members}
		\begin{itemize}
			\item Peer review, editing, and formalization of this document and the knowledge contained herein.
		\end{itemize}
\end{document}

